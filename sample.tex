\documentclass[11pt]{book}

\usepackage[background]{Genesys}

\begin{document}

\chapter{Sample Document}

This is a sample document for the \emph{Genesys} \LaTeX package. Please see below for the various commands.

\section{Dice}

All dice types have their own commands:

\begin{itemize}[noitemsep,nolistsep]
\item \verb|\BoostDie| produces \BoostDie
\item \verb|\AbilityDie| produces \AbilityDie
\item \verb|\ProficiencyDie| produces \ProficiencyDie
\item \verb|\SetbackDie| produces \SetbackDie
\item \verb|\DifficultyDie| produces \DifficultyDie
\item \verb|\ChallengeDie| produces \ChallengeDie
\end{itemize}

\vspace{1em}

And we all know that \BoostDie\AbilityDie\ and \ProficiencyDie give us good symbols: \Advantage\Success\ and \Triumph. Sadly, bad dice give us the negative symbols: \Threat\Failure\ and \Despair.

\section{Tables}

Tables are easy to use with the \verb|GenesysTable| environment.

\begin{GenesysTable}{l X}
Heading & Long Heading\\
\RowColors
Table line one & with the second column\\
And here's & the second line, with blue background!
\end{GenesysTable}

\section{Characters}

When you are making stat blocks for NPCs, be sure to use the \verb|\Characteristics| command which takes 6 arguments, once for each characteristic. Like so:

\vspace{1em}

\Characteristics{1}{3}{2}{2}{2}{2}

\vspace{1em}

Lastly, we have the derived numbers: soak, WT and ST. Use the \verb|\Derived| command, with two arguments?one for the title and the second for the number. For Melee/Ranged defense, we use \verb|\DerivedSplit| with 5 arguments: title, first number, second number, first subtitle and second subtitle. Using \verb|\Derived{Soak}{4}| and \verb|\DerivedSplit{Defense}{2}{0}{Melee}{Ranged}|, for instance, gives us:

\vspace{1em}

\Derived{Soak}{4}\qquad\DerivedSplit{Defense}{2}{0}{Melee}{Ranged}



\end{document}